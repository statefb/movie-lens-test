
% Default to the notebook output style

    


% Inherit from the specified cell style.




    
\documentclass[11pt]{article}

    
    
    \usepackage[T1]{fontenc}
    % Nicer default font (+ math font) than Computer Modern for most use cases
    \usepackage{mathpazo}

    % Basic figure setup, for now with no caption control since it's done
    % automatically by Pandoc (which extracts ![](path) syntax from Markdown).
    \usepackage{graphicx}
    % We will generate all images so they have a width \maxwidth. This means
    % that they will get their normal width if they fit onto the page, but
    % are scaled down if they would overflow the margins.
    \makeatletter
    \def\maxwidth{\ifdim\Gin@nat@width>\linewidth\linewidth
    \else\Gin@nat@width\fi}
    \makeatother
    \let\Oldincludegraphics\includegraphics
    % Set max figure width to be 80% of text width, for now hardcoded.
    \renewcommand{\includegraphics}[1]{\Oldincludegraphics[width=.8\maxwidth]{#1}}
    % Ensure that by default, figures have no caption (until we provide a
    % proper Figure object with a Caption API and a way to capture that
    % in the conversion process - todo).
    \usepackage{caption}
    \DeclareCaptionLabelFormat{nolabel}{}
    \captionsetup{labelformat=nolabel}

    \usepackage{adjustbox} % Used to constrain images to a maximum size 
    \usepackage{xcolor} % Allow colors to be defined
    \usepackage{enumerate} % Needed for markdown enumerations to work
    \usepackage{geometry} % Used to adjust the document margins
    \usepackage{amsmath} % Equations
    \usepackage{amssymb} % Equations
    \usepackage{textcomp} % defines textquotesingle
    % Hack from http://tex.stackexchange.com/a/47451/13684:
    \AtBeginDocument{%
        \def\PYZsq{\textquotesingle}% Upright quotes in Pygmentized code
    }
    \usepackage{upquote} % Upright quotes for verbatim code
    \usepackage{eurosym} % defines \euro
    \usepackage[mathletters]{ucs} % Extended unicode (utf-8) support
    \usepackage[utf8x]{inputenc} % Allow utf-8 characters in the tex document
    \usepackage{fancyvrb} % verbatim replacement that allows latex
    \usepackage{grffile} % extends the file name processing of package graphics 
                         % to support a larger range 
    % The hyperref package gives us a pdf with properly built
    % internal navigation ('pdf bookmarks' for the table of contents,
    % internal cross-reference links, web links for URLs, etc.)
    \usepackage{hyperref}
    \usepackage{longtable} % longtable support required by pandoc >1.10
    \usepackage{booktabs}  % table support for pandoc > 1.12.2
    \usepackage[inline]{enumitem} % IRkernel/repr support (it uses the enumerate* environment)
    \usepackage[normalem]{ulem} % ulem is needed to support strikethroughs (\sout)
                                % normalem makes italics be italics, not underlines
    

    
    
    % Colors for the hyperref package
    \definecolor{urlcolor}{rgb}{0,.145,.698}
    \definecolor{linkcolor}{rgb}{.71,0.21,0.01}
    \definecolor{citecolor}{rgb}{.12,.54,.11}

    % ANSI colors
    \definecolor{ansi-black}{HTML}{3E424D}
    \definecolor{ansi-black-intense}{HTML}{282C36}
    \definecolor{ansi-red}{HTML}{E75C58}
    \definecolor{ansi-red-intense}{HTML}{B22B31}
    \definecolor{ansi-green}{HTML}{00A250}
    \definecolor{ansi-green-intense}{HTML}{007427}
    \definecolor{ansi-yellow}{HTML}{DDB62B}
    \definecolor{ansi-yellow-intense}{HTML}{B27D12}
    \definecolor{ansi-blue}{HTML}{208FFB}
    \definecolor{ansi-blue-intense}{HTML}{0065CA}
    \definecolor{ansi-magenta}{HTML}{D160C4}
    \definecolor{ansi-magenta-intense}{HTML}{A03196}
    \definecolor{ansi-cyan}{HTML}{60C6C8}
    \definecolor{ansi-cyan-intense}{HTML}{258F8F}
    \definecolor{ansi-white}{HTML}{C5C1B4}
    \definecolor{ansi-white-intense}{HTML}{A1A6B2}

    % commands and environments needed by pandoc snippets
    % extracted from the output of `pandoc -s`
    \providecommand{\tightlist}{%
      \setlength{\itemsep}{0pt}\setlength{\parskip}{0pt}}
    \DefineVerbatimEnvironment{Highlighting}{Verbatim}{commandchars=\\\{\}}
    % Add ',fontsize=\small' for more characters per line
    \newenvironment{Shaded}{}{}
    \newcommand{\KeywordTok}[1]{\textcolor[rgb]{0.00,0.44,0.13}{\textbf{{#1}}}}
    \newcommand{\DataTypeTok}[1]{\textcolor[rgb]{0.56,0.13,0.00}{{#1}}}
    \newcommand{\DecValTok}[1]{\textcolor[rgb]{0.25,0.63,0.44}{{#1}}}
    \newcommand{\BaseNTok}[1]{\textcolor[rgb]{0.25,0.63,0.44}{{#1}}}
    \newcommand{\FloatTok}[1]{\textcolor[rgb]{0.25,0.63,0.44}{{#1}}}
    \newcommand{\CharTok}[1]{\textcolor[rgb]{0.25,0.44,0.63}{{#1}}}
    \newcommand{\StringTok}[1]{\textcolor[rgb]{0.25,0.44,0.63}{{#1}}}
    \newcommand{\CommentTok}[1]{\textcolor[rgb]{0.38,0.63,0.69}{\textit{{#1}}}}
    \newcommand{\OtherTok}[1]{\textcolor[rgb]{0.00,0.44,0.13}{{#1}}}
    \newcommand{\AlertTok}[1]{\textcolor[rgb]{1.00,0.00,0.00}{\textbf{{#1}}}}
    \newcommand{\FunctionTok}[1]{\textcolor[rgb]{0.02,0.16,0.49}{{#1}}}
    \newcommand{\RegionMarkerTok}[1]{{#1}}
    \newcommand{\ErrorTok}[1]{\textcolor[rgb]{1.00,0.00,0.00}{\textbf{{#1}}}}
    \newcommand{\NormalTok}[1]{{#1}}
    
    % Additional commands for more recent versions of Pandoc
    \newcommand{\ConstantTok}[1]{\textcolor[rgb]{0.53,0.00,0.00}{{#1}}}
    \newcommand{\SpecialCharTok}[1]{\textcolor[rgb]{0.25,0.44,0.63}{{#1}}}
    \newcommand{\VerbatimStringTok}[1]{\textcolor[rgb]{0.25,0.44,0.63}{{#1}}}
    \newcommand{\SpecialStringTok}[1]{\textcolor[rgb]{0.73,0.40,0.53}{{#1}}}
    \newcommand{\ImportTok}[1]{{#1}}
    \newcommand{\DocumentationTok}[1]{\textcolor[rgb]{0.73,0.13,0.13}{\textit{{#1}}}}
    \newcommand{\AnnotationTok}[1]{\textcolor[rgb]{0.38,0.63,0.69}{\textbf{\textit{{#1}}}}}
    \newcommand{\CommentVarTok}[1]{\textcolor[rgb]{0.38,0.63,0.69}{\textbf{\textit{{#1}}}}}
    \newcommand{\VariableTok}[1]{\textcolor[rgb]{0.10,0.09,0.49}{{#1}}}
    \newcommand{\ControlFlowTok}[1]{\textcolor[rgb]{0.00,0.44,0.13}{\textbf{{#1}}}}
    \newcommand{\OperatorTok}[1]{\textcolor[rgb]{0.40,0.40,0.40}{{#1}}}
    \newcommand{\BuiltInTok}[1]{{#1}}
    \newcommand{\ExtensionTok}[1]{{#1}}
    \newcommand{\PreprocessorTok}[1]{\textcolor[rgb]{0.74,0.48,0.00}{{#1}}}
    \newcommand{\AttributeTok}[1]{\textcolor[rgb]{0.49,0.56,0.16}{{#1}}}
    \newcommand{\InformationTok}[1]{\textcolor[rgb]{0.38,0.63,0.69}{\textbf{\textit{{#1}}}}}
    \newcommand{\WarningTok}[1]{\textcolor[rgb]{0.38,0.63,0.69}{\textbf{\textit{{#1}}}}}
    
    
    % Define a nice break command that doesn't care if a line doesn't already
    % exist.
    \def\br{\hspace*{\fill} \\* }
    % Math Jax compatability definitions
    \def\gt{>}
    \def\lt{<}
    % Document parameters
    \title{report}
    
    
    

    % Pygments definitions
    
\makeatletter
\def\PY@reset{\let\PY@it=\relax \let\PY@bf=\relax%
    \let\PY@ul=\relax \let\PY@tc=\relax%
    \let\PY@bc=\relax \let\PY@ff=\relax}
\def\PY@tok#1{\csname PY@tok@#1\endcsname}
\def\PY@toks#1+{\ifx\relax#1\empty\else%
    \PY@tok{#1}\expandafter\PY@toks\fi}
\def\PY@do#1{\PY@bc{\PY@tc{\PY@ul{%
    \PY@it{\PY@bf{\PY@ff{#1}}}}}}}
\def\PY#1#2{\PY@reset\PY@toks#1+\relax+\PY@do{#2}}

\expandafter\def\csname PY@tok@il\endcsname{\def\PY@tc##1{\textcolor[rgb]{0.40,0.40,0.40}{##1}}}
\expandafter\def\csname PY@tok@nc\endcsname{\let\PY@bf=\textbf\def\PY@tc##1{\textcolor[rgb]{0.00,0.00,1.00}{##1}}}
\expandafter\def\csname PY@tok@m\endcsname{\def\PY@tc##1{\textcolor[rgb]{0.40,0.40,0.40}{##1}}}
\expandafter\def\csname PY@tok@ni\endcsname{\let\PY@bf=\textbf\def\PY@tc##1{\textcolor[rgb]{0.60,0.60,0.60}{##1}}}
\expandafter\def\csname PY@tok@kt\endcsname{\def\PY@tc##1{\textcolor[rgb]{0.69,0.00,0.25}{##1}}}
\expandafter\def\csname PY@tok@sa\endcsname{\def\PY@tc##1{\textcolor[rgb]{0.73,0.13,0.13}{##1}}}
\expandafter\def\csname PY@tok@vg\endcsname{\def\PY@tc##1{\textcolor[rgb]{0.10,0.09,0.49}{##1}}}
\expandafter\def\csname PY@tok@gh\endcsname{\let\PY@bf=\textbf\def\PY@tc##1{\textcolor[rgb]{0.00,0.00,0.50}{##1}}}
\expandafter\def\csname PY@tok@kc\endcsname{\let\PY@bf=\textbf\def\PY@tc##1{\textcolor[rgb]{0.00,0.50,0.00}{##1}}}
\expandafter\def\csname PY@tok@sx\endcsname{\def\PY@tc##1{\textcolor[rgb]{0.00,0.50,0.00}{##1}}}
\expandafter\def\csname PY@tok@gt\endcsname{\def\PY@tc##1{\textcolor[rgb]{0.00,0.27,0.87}{##1}}}
\expandafter\def\csname PY@tok@nb\endcsname{\def\PY@tc##1{\textcolor[rgb]{0.00,0.50,0.00}{##1}}}
\expandafter\def\csname PY@tok@nl\endcsname{\def\PY@tc##1{\textcolor[rgb]{0.63,0.63,0.00}{##1}}}
\expandafter\def\csname PY@tok@sc\endcsname{\def\PY@tc##1{\textcolor[rgb]{0.73,0.13,0.13}{##1}}}
\expandafter\def\csname PY@tok@sb\endcsname{\def\PY@tc##1{\textcolor[rgb]{0.73,0.13,0.13}{##1}}}
\expandafter\def\csname PY@tok@se\endcsname{\let\PY@bf=\textbf\def\PY@tc##1{\textcolor[rgb]{0.73,0.40,0.13}{##1}}}
\expandafter\def\csname PY@tok@cm\endcsname{\let\PY@it=\textit\def\PY@tc##1{\textcolor[rgb]{0.25,0.50,0.50}{##1}}}
\expandafter\def\csname PY@tok@gd\endcsname{\def\PY@tc##1{\textcolor[rgb]{0.63,0.00,0.00}{##1}}}
\expandafter\def\csname PY@tok@k\endcsname{\let\PY@bf=\textbf\def\PY@tc##1{\textcolor[rgb]{0.00,0.50,0.00}{##1}}}
\expandafter\def\csname PY@tok@gi\endcsname{\def\PY@tc##1{\textcolor[rgb]{0.00,0.63,0.00}{##1}}}
\expandafter\def\csname PY@tok@nv\endcsname{\def\PY@tc##1{\textcolor[rgb]{0.10,0.09,0.49}{##1}}}
\expandafter\def\csname PY@tok@w\endcsname{\def\PY@tc##1{\textcolor[rgb]{0.73,0.73,0.73}{##1}}}
\expandafter\def\csname PY@tok@vc\endcsname{\def\PY@tc##1{\textcolor[rgb]{0.10,0.09,0.49}{##1}}}
\expandafter\def\csname PY@tok@err\endcsname{\def\PY@bc##1{\setlength{\fboxsep}{0pt}\fcolorbox[rgb]{1.00,0.00,0.00}{1,1,1}{\strut ##1}}}
\expandafter\def\csname PY@tok@cs\endcsname{\let\PY@it=\textit\def\PY@tc##1{\textcolor[rgb]{0.25,0.50,0.50}{##1}}}
\expandafter\def\csname PY@tok@ss\endcsname{\def\PY@tc##1{\textcolor[rgb]{0.10,0.09,0.49}{##1}}}
\expandafter\def\csname PY@tok@vm\endcsname{\def\PY@tc##1{\textcolor[rgb]{0.10,0.09,0.49}{##1}}}
\expandafter\def\csname PY@tok@s1\endcsname{\def\PY@tc##1{\textcolor[rgb]{0.73,0.13,0.13}{##1}}}
\expandafter\def\csname PY@tok@ow\endcsname{\let\PY@bf=\textbf\def\PY@tc##1{\textcolor[rgb]{0.67,0.13,1.00}{##1}}}
\expandafter\def\csname PY@tok@sh\endcsname{\def\PY@tc##1{\textcolor[rgb]{0.73,0.13,0.13}{##1}}}
\expandafter\def\csname PY@tok@na\endcsname{\def\PY@tc##1{\textcolor[rgb]{0.49,0.56,0.16}{##1}}}
\expandafter\def\csname PY@tok@kd\endcsname{\let\PY@bf=\textbf\def\PY@tc##1{\textcolor[rgb]{0.00,0.50,0.00}{##1}}}
\expandafter\def\csname PY@tok@s\endcsname{\def\PY@tc##1{\textcolor[rgb]{0.73,0.13,0.13}{##1}}}
\expandafter\def\csname PY@tok@gr\endcsname{\def\PY@tc##1{\textcolor[rgb]{1.00,0.00,0.00}{##1}}}
\expandafter\def\csname PY@tok@no\endcsname{\def\PY@tc##1{\textcolor[rgb]{0.53,0.00,0.00}{##1}}}
\expandafter\def\csname PY@tok@mf\endcsname{\def\PY@tc##1{\textcolor[rgb]{0.40,0.40,0.40}{##1}}}
\expandafter\def\csname PY@tok@kp\endcsname{\def\PY@tc##1{\textcolor[rgb]{0.00,0.50,0.00}{##1}}}
\expandafter\def\csname PY@tok@ge\endcsname{\let\PY@it=\textit}
\expandafter\def\csname PY@tok@cp\endcsname{\def\PY@tc##1{\textcolor[rgb]{0.74,0.48,0.00}{##1}}}
\expandafter\def\csname PY@tok@gs\endcsname{\let\PY@bf=\textbf}
\expandafter\def\csname PY@tok@c\endcsname{\let\PY@it=\textit\def\PY@tc##1{\textcolor[rgb]{0.25,0.50,0.50}{##1}}}
\expandafter\def\csname PY@tok@o\endcsname{\def\PY@tc##1{\textcolor[rgb]{0.40,0.40,0.40}{##1}}}
\expandafter\def\csname PY@tok@nd\endcsname{\def\PY@tc##1{\textcolor[rgb]{0.67,0.13,1.00}{##1}}}
\expandafter\def\csname PY@tok@kr\endcsname{\let\PY@bf=\textbf\def\PY@tc##1{\textcolor[rgb]{0.00,0.50,0.00}{##1}}}
\expandafter\def\csname PY@tok@fm\endcsname{\def\PY@tc##1{\textcolor[rgb]{0.00,0.00,1.00}{##1}}}
\expandafter\def\csname PY@tok@s2\endcsname{\def\PY@tc##1{\textcolor[rgb]{0.73,0.13,0.13}{##1}}}
\expandafter\def\csname PY@tok@dl\endcsname{\def\PY@tc##1{\textcolor[rgb]{0.73,0.13,0.13}{##1}}}
\expandafter\def\csname PY@tok@si\endcsname{\let\PY@bf=\textbf\def\PY@tc##1{\textcolor[rgb]{0.73,0.40,0.53}{##1}}}
\expandafter\def\csname PY@tok@go\endcsname{\def\PY@tc##1{\textcolor[rgb]{0.53,0.53,0.53}{##1}}}
\expandafter\def\csname PY@tok@c1\endcsname{\let\PY@it=\textit\def\PY@tc##1{\textcolor[rgb]{0.25,0.50,0.50}{##1}}}
\expandafter\def\csname PY@tok@ch\endcsname{\let\PY@it=\textit\def\PY@tc##1{\textcolor[rgb]{0.25,0.50,0.50}{##1}}}
\expandafter\def\csname PY@tok@vi\endcsname{\def\PY@tc##1{\textcolor[rgb]{0.10,0.09,0.49}{##1}}}
\expandafter\def\csname PY@tok@nt\endcsname{\let\PY@bf=\textbf\def\PY@tc##1{\textcolor[rgb]{0.00,0.50,0.00}{##1}}}
\expandafter\def\csname PY@tok@bp\endcsname{\def\PY@tc##1{\textcolor[rgb]{0.00,0.50,0.00}{##1}}}
\expandafter\def\csname PY@tok@gu\endcsname{\let\PY@bf=\textbf\def\PY@tc##1{\textcolor[rgb]{0.50,0.00,0.50}{##1}}}
\expandafter\def\csname PY@tok@gp\endcsname{\let\PY@bf=\textbf\def\PY@tc##1{\textcolor[rgb]{0.00,0.00,0.50}{##1}}}
\expandafter\def\csname PY@tok@nf\endcsname{\def\PY@tc##1{\textcolor[rgb]{0.00,0.00,1.00}{##1}}}
\expandafter\def\csname PY@tok@mb\endcsname{\def\PY@tc##1{\textcolor[rgb]{0.40,0.40,0.40}{##1}}}
\expandafter\def\csname PY@tok@cpf\endcsname{\let\PY@it=\textit\def\PY@tc##1{\textcolor[rgb]{0.25,0.50,0.50}{##1}}}
\expandafter\def\csname PY@tok@ne\endcsname{\let\PY@bf=\textbf\def\PY@tc##1{\textcolor[rgb]{0.82,0.25,0.23}{##1}}}
\expandafter\def\csname PY@tok@mo\endcsname{\def\PY@tc##1{\textcolor[rgb]{0.40,0.40,0.40}{##1}}}
\expandafter\def\csname PY@tok@kn\endcsname{\let\PY@bf=\textbf\def\PY@tc##1{\textcolor[rgb]{0.00,0.50,0.00}{##1}}}
\expandafter\def\csname PY@tok@mh\endcsname{\def\PY@tc##1{\textcolor[rgb]{0.40,0.40,0.40}{##1}}}
\expandafter\def\csname PY@tok@sd\endcsname{\let\PY@it=\textit\def\PY@tc##1{\textcolor[rgb]{0.73,0.13,0.13}{##1}}}
\expandafter\def\csname PY@tok@mi\endcsname{\def\PY@tc##1{\textcolor[rgb]{0.40,0.40,0.40}{##1}}}
\expandafter\def\csname PY@tok@nn\endcsname{\let\PY@bf=\textbf\def\PY@tc##1{\textcolor[rgb]{0.00,0.00,1.00}{##1}}}
\expandafter\def\csname PY@tok@sr\endcsname{\def\PY@tc##1{\textcolor[rgb]{0.73,0.40,0.53}{##1}}}

\def\PYZbs{\char`\\}
\def\PYZus{\char`\_}
\def\PYZob{\char`\{}
\def\PYZcb{\char`\}}
\def\PYZca{\char`\^}
\def\PYZam{\char`\&}
\def\PYZlt{\char`\<}
\def\PYZgt{\char`\>}
\def\PYZsh{\char`\#}
\def\PYZpc{\char`\%}
\def\PYZdl{\char`\$}
\def\PYZhy{\char`\-}
\def\PYZsq{\char`\'}
\def\PYZdq{\char`\"}
\def\PYZti{\char`\~}
% for compatibility with earlier versions
\def\PYZat{@}
\def\PYZlb{[}
\def\PYZrb{]}
\makeatother


    % Exact colors from NB
    \definecolor{incolor}{rgb}{0.0, 0.0, 0.5}
    \definecolor{outcolor}{rgb}{0.545, 0.0, 0.0}



    
    % Prevent overflowing lines due to hard-to-break entities
    \sloppy 
    % Setup hyperref package
    \hypersetup{
      breaklinks=true,  % so long urls are correctly broken across lines
      colorlinks=true,
      urlcolor=urlcolor,
      linkcolor=linkcolor,
      citecolor=citecolor,
      }
    % Slightly bigger margins than the latex defaults
    
    \geometry{verbose,tmargin=1in,bmargin=1in,lmargin=1in,rmargin=1in}
    
    

    \begin{document}
    
    
    \maketitle
    
    

    
    \section{2017年度リーディングDAT L-B2
課題}\label{ux5e74ux5ea6ux30eaux30fcux30c7ux30a3ux30f3ux30b0dat-l-b2-ux8ab2ux984c}

    \subparagraph{2018/03/11 鈴木 毅洋}\label{ux9234ux6728-ux6bc5ux6d0b}

    \begin{itemize}
\item
  課題設定: 映画のレコメンデーション
\item
  使用したデータ: 映画推薦システム用データセット Movie Lens\\
  下記URLから取得した。\\
  https://github.com/oreilly-japan/ml-at-work\\
  \href{http://yag.xyz/blog/2015/10/03/movielens-datasets/}{概要}:
  \textgreater{}推薦システムの開発やベンチマークのために作られた,映画のレビューためのウェブサイトおよびデータセット.ミネソタ大学のGroupLens
  Researchプロジェクトの一つで,研究目的・非商用でウェブサイトが運用されており,ユーザが好きに映画の情報を眺めたり評価することができる.
\item
  用いた手法: Factorization Machines\\
  協調フィルタリングやNMFに並び、モデルベースのレコメンデーションとして知られるアルゴリズムである。
\end{itemize}

\begin{quote}
FMはSVDやMFとはRatingの持ち方が異なっています。
SVDやMFは1行に1ユーザの評価データを複数格納しており、これをユーザとアイテムの特徴ベクトルをそれぞれ取り出しています。
これに対して、FMは1評価を1行で表し、取り出すのはユーザとアイテムの交互作用(pair-wise)の特徴ベクトルで、SVDやMFとは注目している対象が異なります。
\includegraphics{https://camo.qiitausercontent.com/9a1dcc3a15f83b1bc1da098c6d80f743c145735c/68747470733a2f2f71696974612d696d6167652d73746f72652e73332e616d617a6f6e6177732e636f6d2f302f31323136352f62343537656133662d346232302d323135612d366637642d3162376435363433383564382e706e67}
\end{quote}

なお数式モデルは以下式で与えられる。

\(ŷ (x⃗ )=w_0+∑_{i=1}^{n}w_ix_i+∑_{i=1}^{n}∑_{j=i+1}^n(v⃗ _i⋅v⃗ _j)x_i\)\\
ここで\(w_0,w_i∈ℝ, v⃗ _i∈ℝ\)で、次元rがハイパーパラメータとなる。

上記のように、1行1評価の形でデータを持つことで、User・Item以外の要素(コンテキスト)を加えることが可能であり、\\
より柔軟なモデリングが可能である。

論文:\\
https://www.csie.ntu.edu.tw/\textasciitilde{}b97053/paper/Rendle2010FM.pdf\\
参考ページ:\\
https://qiita.com/wwacky/items/b402a1f3770bee2dd13c\\
http://tech-blog.fancs.com/entry/factorization-machines\\
参考書籍:\\
\href{https://www.oreilly.co.jp/books/9784873118215/}{仕事ではじめる機械学習
オライリー・ジャパン}

    \subsection{EDA (Exploratory Data
Analysis)}\label{eda-exploratory-data-analysis}

    様々な切り口でデータを眺め、概要を把握する。

    \begin{Verbatim}[commandchars=\\\{\}]
{\color{incolor}In [{\color{incolor}1}]:} \PY{o}{\PYZpc{}}\PY{k}{matplotlib} inline
        \PY{k+kn}{import} \PY{n+nn}{pandas} \PY{k}{as} \PY{n+nn}{pd}
        \PY{k+kn}{import} \PY{n+nn}{numpy} \PY{k}{as} \PY{n+nn}{np}
        \PY{k+kn}{import} \PY{n+nn}{matplotlib}\PY{n+nn}{.}\PY{n+nn}{pyplot} \PY{k}{as} \PY{n+nn}{plt}
        \PY{k+kn}{import} \PY{n+nn}{seaborn} \PY{k}{as} \PY{n+nn}{sns}
        \PY{k+kn}{import} \PY{n+nn}{fastFM}
        
        \PY{n}{plt}\PY{o}{.}\PY{n}{style}\PY{o}{.}\PY{n}{use}\PY{p}{(}\PY{l+s+s2}{\PYZdq{}}\PY{l+s+s2}{ggplot}\PY{l+s+s2}{\PYZdq{}}\PY{p}{)}
\end{Verbatim}


    \subsubsection{データの読み込み}\label{ux30c7ux30fcux30bfux306eux8aadux307fux8fbcux307f}

    \begin{Verbatim}[commandchars=\\\{\}]
{\color{incolor}In [{\color{incolor}3}]:} \PY{c+c1}{\PYZsh{} ユーザー情報}
        \PY{n}{u\PYZus{}cols} \PY{o}{=} \PY{p}{[}\PY{l+s+s2}{\PYZdq{}}\PY{l+s+s2}{user\PYZus{}id}\PY{l+s+s2}{\PYZdq{}}\PY{p}{,}\PY{l+s+s2}{\PYZdq{}}\PY{l+s+s2}{age}\PY{l+s+s2}{\PYZdq{}}\PY{p}{,}\PY{l+s+s2}{\PYZdq{}}\PY{l+s+s2}{sex}\PY{l+s+s2}{\PYZdq{}}\PY{p}{,}\PY{l+s+s2}{\PYZdq{}}\PY{l+s+s2}{occupation}\PY{l+s+s2}{\PYZdq{}}\PY{p}{,}\PY{l+s+s2}{\PYZdq{}}\PY{l+s+s2}{zip\PYZhy{}code}\PY{l+s+s2}{\PYZdq{}}\PY{p}{]}
        \PY{n}{users} \PY{o}{=} \PY{n}{pd}\PY{o}{.}\PY{n}{read\PYZus{}csv}\PY{p}{(}\PY{l+s+s2}{\PYZdq{}}\PY{l+s+s2}{../data/ml\PYZhy{}100k/u.user}\PY{l+s+s2}{\PYZdq{}}\PY{p}{,}\PY{n}{sep}\PY{o}{=}\PY{l+s+s2}{\PYZdq{}}\PY{l+s+s2}{|}\PY{l+s+s2}{\PYZdq{}}\PY{p}{,}\PY{n}{names}\PY{o}{=}\PY{n}{u\PYZus{}cols}\PY{p}{)}
        
        \PY{c+c1}{\PYZsh{} 評価値情報}
        \PY{n}{r\PYZus{}cols} \PY{o}{=} \PY{p}{[}\PY{l+s+s2}{\PYZdq{}}\PY{l+s+s2}{user\PYZus{}id}\PY{l+s+s2}{\PYZdq{}}\PY{p}{,}\PY{l+s+s2}{\PYZdq{}}\PY{l+s+s2}{movie\PYZus{}id}\PY{l+s+s2}{\PYZdq{}}\PY{p}{,}\PY{l+s+s2}{\PYZdq{}}\PY{l+s+s2}{rating}\PY{l+s+s2}{\PYZdq{}}\PY{p}{,}\PY{l+s+s2}{\PYZdq{}}\PY{l+s+s2}{unix\PYZus{}timestamp}\PY{l+s+s2}{\PYZdq{}}\PY{p}{]}
        \PY{n}{ratings} \PY{o}{=} \PY{n}{pd}\PY{o}{.}\PY{n}{read\PYZus{}csv}\PY{p}{(}\PY{l+s+s2}{\PYZdq{}}\PY{l+s+s2}{../data/ml\PYZhy{}100k/u.data}\PY{l+s+s2}{\PYZdq{}}\PY{p}{,}\PY{n}{sep}\PY{o}{=}\PY{l+s+s2}{\PYZdq{}}\PY{l+s+se}{\PYZbs{}t}\PY{l+s+s2}{\PYZdq{}}\PY{p}{,}\PY{n}{names}\PY{o}{=}\PY{n}{r\PYZus{}cols}\PY{p}{)}
        \PY{n}{ratings}\PY{p}{[}\PY{l+s+s2}{\PYZdq{}}\PY{l+s+s2}{date}\PY{l+s+s2}{\PYZdq{}}\PY{p}{]} \PY{o}{=} \PY{n}{pd}\PY{o}{.}\PY{n}{to\PYZus{}datetime}\PY{p}{(}\PY{n}{ratings}\PY{o}{.}\PY{n}{unix\PYZus{}timestamp}\PY{p}{,}\PY{n}{unit}\PY{o}{=}\PY{l+s+s2}{\PYZdq{}}\PY{l+s+s2}{s}\PY{l+s+s2}{\PYZdq{}}\PY{p}{)}
        
        \PY{c+c1}{\PYZsh{} 映画情報}
        \PY{n}{m\PYZus{}cols} \PY{o}{=} \PY{p}{[}\PY{l+s+s2}{\PYZdq{}}\PY{l+s+s2}{movie\PYZus{}id}\PY{l+s+s2}{\PYZdq{}}\PY{p}{,}\PY{l+s+s2}{\PYZdq{}}\PY{l+s+s2}{title}\PY{l+s+s2}{\PYZdq{}}\PY{p}{,}\PY{l+s+s2}{\PYZdq{}}\PY{l+s+s2}{release\PYZus{}date}\PY{l+s+s2}{\PYZdq{}}\PY{p}{,}\PY{l+s+s2}{\PYZdq{}}\PY{l+s+s2}{video\PYZus{}release\PYZus{}date}\PY{l+s+s2}{\PYZdq{}}\PY{p}{,}\PY{l+s+s2}{\PYZdq{}}\PY{l+s+s2}{imdb\PYZus{}url}\PY{l+s+s2}{\PYZdq{}}\PY{p}{]}
        \PY{n}{movies} \PY{o}{=} \PY{n}{pd}\PY{o}{.}\PY{n}{read\PYZus{}csv}\PY{p}{(}\PY{l+s+s2}{\PYZdq{}}\PY{l+s+s2}{../data/ml\PYZhy{}100k/u.item}\PY{l+s+s2}{\PYZdq{}}\PY{p}{,}\PY{n}{sep}\PY{o}{=}\PY{l+s+s2}{\PYZdq{}}\PY{l+s+s2}{|}\PY{l+s+s2}{\PYZdq{}}\PY{p}{,}\PY{n}{names}\PY{o}{=}\PY{n}{m\PYZus{}cols}\PY{p}{,}\PY{n}{usecols}\PY{o}{=}\PY{n+nb}{range}\PY{p}{(}\PY{l+m+mi}{5}\PY{p}{)}\PY{p}{,}\PY{n}{encoding}\PY{o}{=}\PY{l+s+s2}{\PYZdq{}}\PY{l+s+s2}{latin1}\PY{l+s+s2}{\PYZdq{}}\PY{p}{)}
        
        \PY{c+c1}{\PYZsh{} 全てのデータをマージ}
        \PY{n}{movie\PYZus{}rating} \PY{o}{=} \PY{n}{pd}\PY{o}{.}\PY{n}{merge}\PY{p}{(}\PY{n}{movies}\PY{p}{,}\PY{n}{ratings}\PY{p}{)}
        \PY{n}{lens} \PY{o}{=} \PY{n}{pd}\PY{o}{.}\PY{n}{merge}\PY{p}{(}\PY{n}{movie\PYZus{}rating}\PY{p}{,}\PY{n}{users}\PY{p}{)}
\end{Verbatim}


    \begin{Verbatim}[commandchars=\\\{\}]
{\color{incolor}In [{\color{incolor}7}]:} \PY{n}{lens}\PY{o}{.}\PY{n}{head}\PY{p}{(}\PY{p}{)}
\end{Verbatim}


\begin{Verbatim}[commandchars=\\\{\}]
{\color{outcolor}Out[{\color{outcolor}7}]:}    movie\_id                  title release\_date  video\_release\_date  \textbackslash{}
        0         1       Toy Story (1995)  01-Jan-1995                 NaN   
        1         4      Get Shorty (1995)  01-Jan-1995                 NaN   
        2         5         Copycat (1995)  01-Jan-1995                 NaN   
        3         7  Twelve Monkeys (1995)  01-Jan-1995                 NaN   
        4         8            Babe (1995)  01-Jan-1995                 NaN   
        
                                                    imdb\_url  user\_id  rating  \textbackslash{}
        0  http://us.imdb.com/M/title-exact?Toy\%20Story\%2{\ldots}      308       4   
        1  http://us.imdb.com/M/title-exact?Get\%20Shorty\%{\ldots}      308       5   
        2  http://us.imdb.com/M/title-exact?Copycat\%20(1995)      308       4   
        3  http://us.imdb.com/M/title-exact?Twelve\%20Monk{\ldots}      308       4   
        4     http://us.imdb.com/M/title-exact?Babe\%20(1995)      308       5   
        
           unix\_timestamp                date  age sex occupation zip-code  
        0       887736532 1998-02-17 17:28:52   60   M    retired    95076  
        1       887737890 1998-02-17 17:51:30   60   M    retired    95076  
        2       887739608 1998-02-17 18:20:08   60   M    retired    95076  
        3       887738847 1998-02-17 18:07:27   60   M    retired    95076  
        4       887736696 1998-02-17 17:31:36   60   M    retired    95076  
\end{Verbatim}
            
    \subsubsection{Summary}\label{summary}

    \begin{Verbatim}[commandchars=\\\{\}]
{\color{incolor}In [{\color{incolor}8}]:} \PY{c+c1}{\PYZsh{} 最も評価された回数の多い25作品のタイトル}
        \PY{n}{lens}\PY{o}{.}\PY{n}{title}\PY{o}{.}\PY{n}{value\PYZus{}counts}\PY{p}{(}\PY{p}{)}\PY{o}{.}\PY{n}{head}\PY{p}{(}\PY{l+m+mi}{25}\PY{p}{)}
\end{Verbatim}


\begin{Verbatim}[commandchars=\\\{\}]
{\color{outcolor}Out[{\color{outcolor}8}]:} Star Wars (1977)                             583
        Contact (1997)                               509
        Fargo (1996)                                 508
        Return of the Jedi (1983)                    507
        Liar Liar (1997)                             485
        English Patient, The (1996)                  481
        Scream (1996)                                478
        Toy Story (1995)                             452
        Air Force One (1997)                         431
        Independence Day (ID4) (1996)                429
        Raiders of the Lost Ark (1981)               420
        Godfather, The (1972)                        413
        Pulp Fiction (1994)                          394
        Twelve Monkeys (1995)                        392
        Silence of the Lambs, The (1991)             390
        Jerry Maguire (1996)                         384
        Chasing Amy (1997)                           379
        Rock, The (1996)                             378
        Empire Strikes Back, The (1980)              367
        Star Trek: First Contact (1996)              365
        Back to the Future (1985)                    350
        Titanic (1997)                               350
        Mission: Impossible (1996)                   344
        Fugitive, The (1993)                         336
        Indiana Jones and the Last Crusade (1989)    331
        Name: title, dtype: int64
\end{Verbatim}
            
    \begin{itemize}
\tightlist
\item
  2000年以前の映画が多い。古い映画の方が評価人数が多いため?\\
  →映画ごとに評価の数と平均を集計し、平均値の高い順に並べてみる
\end{itemize}

    \begin{Verbatim}[commandchars=\\\{\}]
{\color{incolor}In [{\color{incolor}9}]:} \PY{c+c1}{\PYZsh{} 100件以上評価されているタイトルのみを対象}
        \PY{n}{atleast\PYZus{}100} \PY{o}{=} \PY{n}{movie\PYZus{}stats}\PY{p}{[}\PY{l+s+s2}{\PYZdq{}}\PY{l+s+s2}{rating}\PY{l+s+s2}{\PYZdq{}}\PY{p}{]}\PY{p}{[}\PY{l+s+s2}{\PYZdq{}}\PY{l+s+s2}{size}\PY{l+s+s2}{\PYZdq{}}\PY{p}{]} \PY{o}{\PYZgt{}}\PY{o}{=} \PY{l+m+mi}{100}
        \PY{n}{movie\PYZus{}stats}\PY{p}{[}\PY{n}{atleast\PYZus{}100}\PY{p}{]}\PY{o}{.}\PY{n}{sort\PYZus{}values}\PY{p}{(}\PY{n}{by}\PY{o}{=}\PY{p}{[}\PY{p}{(}\PY{l+s+s2}{\PYZdq{}}\PY{l+s+s2}{rating}\PY{l+s+s2}{\PYZdq{}}\PY{p}{,}\PY{l+s+s2}{\PYZdq{}}\PY{l+s+s2}{mean}\PY{l+s+s2}{\PYZdq{}}\PY{p}{)}\PY{p}{]}\PY{p}{,}\PY{n}{ascending}\PY{o}{=}\PY{k+kc}{False}\PY{p}{)}\PY{o}{.}\PY{n}{head}\PY{p}{(}\PY{l+m+mi}{15}\PY{p}{)}
\end{Verbatim}


\begin{Verbatim}[commandchars=\\\{\}]
{\color{outcolor}Out[{\color{outcolor}9}]:}                                        rating          
                                                 size      mean
        title                                                  
        Close Shave, A (1995)                     112  4.491071
        Schindler's List (1993)                   298  4.466443
        Wrong Trousers, The (1993)                118  4.466102
        Casablanca (1942)                         243  4.456790
        Shawshank Redemption, The (1994)          283  4.445230
        Rear Window (1954)                        209  4.387560
        Usual Suspects, The (1995)                267  4.385768
        Star Wars (1977)                          583  4.358491
        12 Angry Men (1957)                       125  4.344000
        Citizen Kane (1941)                       198  4.292929
        To Kill a Mockingbird (1962)              219  4.292237
        One Flew Over the Cuckoo's Nest (1975)    264  4.291667
        Silence of the Lambs, The (1991)          390  4.289744
        North by Northwest (1959)                 179  4.284916
        Godfather, The (1972)                     413  4.283293
\end{Verbatim}
            
    \begin{Verbatim}[commandchars=\\\{\}]
{\color{incolor}In [{\color{incolor}10}]:} \PY{c+c1}{\PYZsh{} 評価回数の分布}
         \PY{n}{lens}\PY{o}{.}\PY{n}{groupby}\PY{p}{(}\PY{l+s+s2}{\PYZdq{}}\PY{l+s+s2}{user\PYZus{}id}\PY{l+s+s2}{\PYZdq{}}\PY{p}{)}\PY{o}{.}\PY{n}{size}\PY{p}{(}\PY{p}{)}\PY{o}{.}\PY{n}{sort\PYZus{}values}\PY{p}{(}\PY{n}{ascending}\PY{o}{=}\PY{k+kc}{False}\PY{p}{)}\PY{o}{.}\PY{n}{hist}\PY{p}{(}\PY{p}{)}
         \PY{n}{plt}\PY{o}{.}\PY{n}{xlabel}\PY{p}{(}\PY{l+s+s2}{\PYZdq{}}\PY{l+s+s2}{rating size}\PY{l+s+s2}{\PYZdq{}}\PY{p}{)}
         \PY{n}{plt}\PY{o}{.}\PY{n}{ylabel}\PY{p}{(}\PY{l+s+s2}{\PYZdq{}}\PY{l+s+s2}{count of rating}\PY{l+s+s2}{\PYZdq{}}\PY{p}{)}
\end{Verbatim}


\begin{Verbatim}[commandchars=\\\{\}]
{\color{outcolor}Out[{\color{outcolor}10}]:} <matplotlib.text.Text at 0x11ae44e48>
\end{Verbatim}
            
    \begin{center}
    \adjustimage{max size={0.9\linewidth}{0.9\paperheight}}{output_13_1.png}
    \end{center}
    { \hspace*{\fill} \\}
    
    \begin{itemize}
\tightlist
\item
  沢山のタイトルを評価しているユーザーは少なく、逆も然りである。\\
  典型的なロングテールの分布であることがわかる
\end{itemize}

    \begin{Verbatim}[commandchars=\\\{\}]
{\color{incolor}In [{\color{incolor}12}]:} \PY{c+c1}{\PYZsh{} ユーザーごとの評価数と評価値の平均}
         \PY{n}{user\PYZus{}stats} \PY{o}{=} \PY{n}{lens}\PY{o}{.}\PY{n}{groupby}\PY{p}{(}\PY{l+s+s2}{\PYZdq{}}\PY{l+s+s2}{user\PYZus{}id}\PY{l+s+s2}{\PYZdq{}}\PY{p}{)}\PY{o}{.}\PY{n}{agg}\PY{p}{(}\PY{n+nb}{dict}\PY{p}{(}\PY{n}{rating}\PY{o}{=}\PY{p}{[}\PY{n}{np}\PY{o}{.}\PY{n}{size}\PY{p}{,}\PY{n}{np}\PY{o}{.}\PY{n}{mean}\PY{p}{]}\PY{p}{)}\PY{p}{)}
         \PY{n}{user\PYZus{}stats}\PY{o}{.}\PY{n}{rating}\PY{o}{.}\PY{n}{describe}\PY{p}{(}\PY{p}{)}
\end{Verbatim}


\begin{Verbatim}[commandchars=\\\{\}]
{\color{outcolor}Out[{\color{outcolor}12}]:}              size        mean
         count  943.000000  943.000000
         mean   106.044539    3.588191
         std    100.931743    0.445233
         min     20.000000    1.491954
         25\%     33.000000    3.323054
         50\%     65.000000    3.620690
         75\%    148.000000    3.869565
         max    737.000000    4.869565
\end{Verbatim}
            
    \begin{itemize}
\tightlist
\item
  評価値の平均が1.49と辛口のユーザーもいれば、4.87と甘口のユーザーも居る
\end{itemize}

    \subsection{Factorization
Machinsによるレコメンデーション}\label{factorization-machinsux306bux3088ux308bux30ecux30b3ux30e1ux30f3ux30c7ux30fcux30b7ux30e7ux30f3}

    \begin{itemize}
\tightlist
\item
  提案者の実装した\href{http://www.libfm.org/}{libFMというライブラリ}を使用\\
  (実際にはPythonでラップしたfastFMを使用)
\item
  ALS, SGD, MCMCの3種の実装が存在し、場合に応じて使い分けることが可能
\end{itemize}

    \begin{Verbatim}[commandchars=\\\{\}]
{\color{incolor}In [{\color{incolor}17}]:} \PY{k+kn}{from} \PY{n+nn}{fastFM} \PY{k}{import} \PY{n}{mcmc}
         \PY{k+kn}{from} \PY{n+nn}{sklearn}\PY{n+nn}{.}\PY{n+nn}{metrics} \PY{k}{import} \PY{n}{mean\PYZus{}squared\PYZus{}error}
         \PY{k+kn}{from} \PY{n+nn}{sklearn}\PY{n+nn}{.}\PY{n+nn}{preprocessing} \PY{k}{import} \PY{n}{StandardScaler}
         \PY{k+kn}{from} \PY{n+nn}{sklearn}\PY{n+nn}{.}\PY{n+nn}{feature\PYZus{}extraction} \PY{k}{import} \PY{n}{DictVectorizer}
         \PY{k+kn}{from} \PY{n+nn}{sklearn}\PY{n+nn}{.}\PY{n+nn}{model\PYZus{}selection} \PY{k}{import} \PY{n}{train\PYZus{}test\PYZus{}split}
         \PY{k+kn}{from} \PY{n+nn}{tqdm} \PY{k}{import} \PY{n}{tqdm}
\end{Verbatim}


    \begin{Verbatim}[commandchars=\\\{\}]
{\color{incolor}In [{\color{incolor}14}]:} \PY{c+c1}{\PYZsh{} 前処理}
         \PY{n}{lens}\PY{p}{[}\PY{l+s+s2}{\PYZdq{}}\PY{l+s+s2}{user\PYZus{}id}\PY{l+s+s2}{\PYZdq{}}\PY{p}{]} \PY{o}{=} \PY{n}{lens}\PY{p}{[}\PY{l+s+s2}{\PYZdq{}}\PY{l+s+s2}{user\PYZus{}id}\PY{l+s+s2}{\PYZdq{}}\PY{p}{]}\PY{o}{.}\PY{n}{astype}\PY{p}{(}\PY{n+nb}{str}\PY{p}{)}
         \PY{n}{lens}\PY{p}{[}\PY{l+s+s2}{\PYZdq{}}\PY{l+s+s2}{movie\PYZus{}id}\PY{l+s+s2}{\PYZdq{}}\PY{p}{]} \PY{o}{=} \PY{n}{lens}\PY{p}{[}\PY{l+s+s2}{\PYZdq{}}\PY{l+s+s2}{movie\PYZus{}id}\PY{l+s+s2}{\PYZdq{}}\PY{p}{]}\PY{o}{.}\PY{n}{astype}\PY{p}{(}\PY{n+nb}{str}\PY{p}{)}
         \PY{n}{lens}\PY{p}{[}\PY{l+s+s2}{\PYZdq{}}\PY{l+s+s2}{year}\PY{l+s+s2}{\PYZdq{}}\PY{p}{]} \PY{o}{=} \PY{n}{lens}\PY{p}{[}\PY{l+s+s2}{\PYZdq{}}\PY{l+s+s2}{date}\PY{l+s+s2}{\PYZdq{}}\PY{p}{]}\PY{o}{.}\PY{n}{apply}\PY{p}{(}\PY{n+nb}{str}\PY{p}{)}\PY{o}{.}\PY{n}{str}\PY{o}{.}\PY{n}{split}\PY{p}{(}\PY{l+s+s2}{\PYZdq{}}\PY{l+s+s2}{\PYZhy{}}\PY{l+s+s2}{\PYZdq{}}\PY{p}{)}\PY{o}{.}\PY{n}{str}\PY{o}{.}\PY{n}{get}\PY{p}{(}\PY{l+m+mi}{0}\PY{p}{)}
         \PY{n}{lens}\PY{p}{[}\PY{l+s+s2}{\PYZdq{}}\PY{l+s+s2}{release\PYZus{}year}\PY{l+s+s2}{\PYZdq{}}\PY{p}{]} \PY{o}{=} \PY{n}{lens}\PY{p}{[}\PY{l+s+s2}{\PYZdq{}}\PY{l+s+s2}{release\PYZus{}date}\PY{l+s+s2}{\PYZdq{}}\PY{p}{]}\PY{o}{.}\PY{n}{apply}\PY{p}{(}\PY{n+nb}{str}\PY{p}{)}\PY{o}{.}\PY{n}{str}\PY{o}{.}\PY{n}{split}\PY{p}{(}\PY{l+s+s2}{\PYZdq{}}\PY{l+s+s2}{\PYZhy{}}\PY{l+s+s2}{\PYZdq{}}\PY{p}{)}\PY{o}{.}\PY{n}{str}\PY{o}{.}\PY{n}{get}\PY{p}{(}\PY{l+m+mi}{2}\PY{p}{)}
         \PY{n}{lens}\PY{p}{[}\PY{l+s+s2}{\PYZdq{}}\PY{l+s+s2}{year}\PY{l+s+s2}{\PYZdq{}}\PY{p}{]} \PY{o}{=} \PY{n}{lens}\PY{p}{[}\PY{l+s+s2}{\PYZdq{}}\PY{l+s+s2}{date}\PY{l+s+s2}{\PYZdq{}}\PY{p}{]}\PY{o}{.}\PY{n}{apply}\PY{p}{(}\PY{n+nb}{str}\PY{p}{)}\PY{o}{.}\PY{n}{str}\PY{o}{.}\PY{n}{split}\PY{p}{(}\PY{l+s+s2}{\PYZdq{}}\PY{l+s+s2}{\PYZhy{}}\PY{l+s+s2}{\PYZdq{}}\PY{p}{)}\PY{o}{.}\PY{n}{str}\PY{o}{.}\PY{n}{get}\PY{p}{(}\PY{l+m+mi}{0}\PY{p}{)}
         \PY{n}{lens}\PY{p}{[}\PY{l+s+s2}{\PYZdq{}}\PY{l+s+s2}{release\PYZus{}year}\PY{l+s+s2}{\PYZdq{}}\PY{p}{]} \PY{o}{=} \PY{n}{lens}\PY{p}{[}\PY{l+s+s2}{\PYZdq{}}\PY{l+s+s2}{release\PYZus{}date}\PY{l+s+s2}{\PYZdq{}}\PY{p}{]}\PY{o}{.}\PY{n}{apply}\PY{p}{(}\PY{n+nb}{str}\PY{p}{)}\PY{o}{.}\PY{n}{str}\PY{o}{.}\PY{n}{split}\PY{p}{(}\PY{l+s+s2}{\PYZdq{}}\PY{l+s+s2}{\PYZhy{}}\PY{l+s+s2}{\PYZdq{}}\PY{p}{)}\PY{o}{.}\PY{n}{str}\PY{o}{.}\PY{n}{get}\PY{p}{(}\PY{l+m+mi}{2}\PY{p}{)}
\end{Verbatim}


    \begin{itemize}
\tightlist
\item
  下記の複数の特徴量の組み合わせの中から、最も良いものを選択する
\end{itemize}

    \begin{Verbatim}[commandchars=\\\{\}]
{\color{incolor}In [{\color{incolor}15}]:} \PY{c+c1}{\PYZsh{} 特徴量の組み合わせ候補}
         \PY{n}{candidate\PYZus{}columns} \PY{o}{=} \PY{p}{[}
             \PY{p}{[}\PY{l+s+s2}{\PYZdq{}}\PY{l+s+s2}{user\PYZus{}id}\PY{l+s+s2}{\PYZdq{}}\PY{p}{,}\PY{l+s+s2}{\PYZdq{}}\PY{l+s+s2}{movie\PYZus{}id}\PY{l+s+s2}{\PYZdq{}}\PY{p}{,}\PY{l+s+s2}{\PYZdq{}}\PY{l+s+s2}{release\PYZus{}year}\PY{l+s+s2}{\PYZdq{}}\PY{p}{,}\PY{l+s+s2}{\PYZdq{}}\PY{l+s+s2}{age}\PY{l+s+s2}{\PYZdq{}}\PY{p}{,}\PY{l+s+s2}{\PYZdq{}}\PY{l+s+s2}{sex}\PY{l+s+s2}{\PYZdq{}}\PY{p}{,}\PY{l+s+s2}{\PYZdq{}}\PY{l+s+s2}{year}\PY{l+s+s2}{\PYZdq{}}\PY{p}{,}\PY{l+s+s2}{\PYZdq{}}\PY{l+s+s2}{rating}\PY{l+s+s2}{\PYZdq{}}\PY{p}{]}\PY{p}{,}  \PY{c+c1}{\PYZsh{}A}
             \PY{p}{[}\PY{l+s+s2}{\PYZdq{}}\PY{l+s+s2}{user\PYZus{}id}\PY{l+s+s2}{\PYZdq{}}\PY{p}{,}\PY{l+s+s2}{\PYZdq{}}\PY{l+s+s2}{movie\PYZus{}id}\PY{l+s+s2}{\PYZdq{}}\PY{p}{,}\PY{l+s+s2}{\PYZdq{}}\PY{l+s+s2}{age}\PY{l+s+s2}{\PYZdq{}}\PY{p}{,}\PY{l+s+s2}{\PYZdq{}}\PY{l+s+s2}{sex}\PY{l+s+s2}{\PYZdq{}}\PY{p}{,}\PY{l+s+s2}{\PYZdq{}}\PY{l+s+s2}{year}\PY{l+s+s2}{\PYZdq{}}\PY{p}{,}\PY{l+s+s2}{\PYZdq{}}\PY{l+s+s2}{rating}\PY{l+s+s2}{\PYZdq{}}\PY{p}{]}\PY{p}{,}  \PY{c+c1}{\PYZsh{}B}
             \PY{p}{[}\PY{l+s+s2}{\PYZdq{}}\PY{l+s+s2}{user\PYZus{}id}\PY{l+s+s2}{\PYZdq{}}\PY{p}{,}\PY{l+s+s2}{\PYZdq{}}\PY{l+s+s2}{movie\PYZus{}id}\PY{l+s+s2}{\PYZdq{}}\PY{p}{,}\PY{l+s+s2}{\PYZdq{}}\PY{l+s+s2}{sex}\PY{l+s+s2}{\PYZdq{}}\PY{p}{,}\PY{l+s+s2}{\PYZdq{}}\PY{l+s+s2}{year}\PY{l+s+s2}{\PYZdq{}}\PY{p}{,}\PY{l+s+s2}{\PYZdq{}}\PY{l+s+s2}{rating}\PY{l+s+s2}{\PYZdq{}}\PY{p}{]}\PY{p}{,}  \PY{c+c1}{\PYZsh{}C}
             \PY{p}{[}\PY{l+s+s2}{\PYZdq{}}\PY{l+s+s2}{user\PYZus{}id}\PY{l+s+s2}{\PYZdq{}}\PY{p}{,}\PY{l+s+s2}{\PYZdq{}}\PY{l+s+s2}{movie\PYZus{}id}\PY{l+s+s2}{\PYZdq{}}\PY{p}{,}\PY{l+s+s2}{\PYZdq{}}\PY{l+s+s2}{age}\PY{l+s+s2}{\PYZdq{}}\PY{p}{,}\PY{l+s+s2}{\PYZdq{}}\PY{l+s+s2}{sex}\PY{l+s+s2}{\PYZdq{}}\PY{p}{,}\PY{l+s+s2}{\PYZdq{}}\PY{l+s+s2}{rating}\PY{l+s+s2}{\PYZdq{}}\PY{p}{]}\PY{p}{,}   \PY{c+c1}{\PYZsh{}D}
             \PY{p}{[}\PY{l+s+s2}{\PYZdq{}}\PY{l+s+s2}{user\PYZus{}id}\PY{l+s+s2}{\PYZdq{}}\PY{p}{,}\PY{l+s+s2}{\PYZdq{}}\PY{l+s+s2}{movie\PYZus{}id}\PY{l+s+s2}{\PYZdq{}}\PY{p}{,}\PY{l+s+s2}{\PYZdq{}}\PY{l+s+s2}{rating}\PY{l+s+s2}{\PYZdq{}}\PY{p}{]}\PY{p}{,}  \PY{c+c1}{\PYZsh{}E}
         \PY{p}{]}
\end{Verbatim}


    \begin{Verbatim}[commandchars=\\\{\}]
{\color{incolor}In [{\color{incolor}19}]:} \PY{n}{rmse\PYZus{}test} \PY{o}{=} \PY{p}{[}\PY{p}{]}
         
         \PY{k}{for} \PY{n}{column} \PY{o+ow}{in} \PY{n}{candidate\PYZus{}columns}\PY{p}{:}
             \PY{c+c1}{\PYZsh{} 欠損値削除}
             \PY{n}{filtered\PYZus{}lens} \PY{o}{=} \PY{n}{lens}\PY{p}{[}\PY{n}{column}\PY{p}{]}\PY{o}{.}\PY{n}{dropna}\PY{p}{(}\PY{p}{)}
             \PY{c+c1}{\PYZsh{} 入力データをダミー変数に変換}
             \PY{n}{v} \PY{o}{=} \PY{n}{DictVectorizer}\PY{p}{(}\PY{p}{)}
             \PY{n}{X\PYZus{}more\PYZus{}feature} \PY{o}{=} \PY{n}{v}\PY{o}{.}\PY{n}{fit\PYZus{}transform}\PY{p}{(}
                 \PY{n+nb}{list}\PY{p}{(}\PY{n}{filtered\PYZus{}lens}\PY{o}{.}\PY{n}{drop}\PY{p}{(}\PY{l+s+s2}{\PYZdq{}}\PY{l+s+s2}{rating}\PY{l+s+s2}{\PYZdq{}}\PY{p}{,}\PY{n}{axis}\PY{o}{=}\PY{l+m+mi}{1}\PY{p}{)}\PY{o}{.}\PY{n}{T}\PY{o}{.}\PY{n}{to\PYZus{}dict}\PY{p}{(}\PY{p}{)}\PY{o}{.}\PY{n}{values}\PY{p}{(}\PY{p}{)}\PY{p}{)}\PY{p}{)}
             \PY{n}{y\PYZus{}more\PYZus{}feature} \PY{o}{=} \PY{n}{filtered\PYZus{}lens}\PY{p}{[}\PY{l+s+s2}{\PYZdq{}}\PY{l+s+s2}{rating}\PY{l+s+s2}{\PYZdq{}}\PY{p}{]}\PY{o}{.}\PY{n}{tolist}\PY{p}{(}\PY{p}{)}
             
             \PY{c+c1}{\PYZsh{} 教師データの学習用と評価用の分割}
             \PY{n}{X\PYZus{}mf\PYZus{}train}\PY{p}{,}\PY{n}{X\PYZus{}mf\PYZus{}test}\PY{p}{,}\PY{n}{y\PYZus{}mf\PYZus{}train}\PY{p}{,}\PY{n}{y\PYZus{}mf\PYZus{}test} \PY{o}{=} \PY{n}{train\PYZus{}test\PYZus{}split}\PY{p}{(}
                 \PY{n}{X\PYZus{}more\PYZus{}feature}\PY{p}{,}\PY{n}{y\PYZus{}more\PYZus{}feature}\PY{p}{,}\PY{n}{test\PYZus{}size}\PY{o}{=}\PY{o}{.}\PY{l+m+mi}{1}\PY{p}{,}\PY{n}{random\PYZus{}state}\PY{o}{=}\PY{l+m+mi}{42}\PY{p}{)}
             
             \PY{c+c1}{\PYZsh{} ratingの正規化}
             \PY{n}{scaler} \PY{o}{=} \PY{n}{StandardScaler}\PY{p}{(}\PY{p}{)}
             \PY{n}{y\PYZus{}mf\PYZus{}train\PYZus{}norm} \PY{o}{=} \PY{n}{scaler}\PY{o}{.}\PY{n}{fit\PYZus{}transform}\PY{p}{(}\PY{n}{np}\PY{o}{.}\PY{n}{array}\PY{p}{(}\PY{n}{y\PYZus{}mf\PYZus{}train}\PY{p}{)}\PY{o}{.}\PY{n}{reshape}\PY{p}{(}\PY{o}{\PYZhy{}}\PY{l+m+mi}{1}\PY{p}{,} \PY{l+m+mi}{1}\PY{p}{)}\PY{p}{)}\PY{o}{.}\PY{n}{ravel}\PY{p}{(}\PY{p}{)}
             
             \PY{c+c1}{\PYZsh{} MCMCを使ったモデルの学習}
             \PY{n}{fm} \PY{o}{=} \PY{n}{mcmc}\PY{o}{.}\PY{n}{FMRegression}\PY{p}{(}\PY{n}{n\PYZus{}iter}\PY{o}{=}\PY{l+m+mi}{500}\PY{p}{,}\PY{n}{rank}\PY{o}{=}\PY{l+m+mi}{8}\PY{p}{,}\PY{n}{random\PYZus{}state}\PY{o}{=}\PY{l+m+mi}{123}\PY{p}{)}
             \PY{n}{fm}\PY{o}{.}\PY{n}{fit\PYZus{}predict}\PY{p}{(}\PY{n}{X\PYZus{}mf\PYZus{}train}\PY{p}{,}\PY{n}{y\PYZus{}mf\PYZus{}train\PYZus{}norm}\PY{p}{,}\PY{n}{X\PYZus{}mf\PYZus{}test}\PY{p}{)}
             
             \PY{c+c1}{\PYZsh{} テストデータでの予測結果のRMSE取得}
             \PY{n}{y\PYZus{}pred} \PY{o}{=} \PY{n}{fm}\PY{o}{.}\PY{n}{fit\PYZus{}predict}\PY{p}{(}\PY{n}{X\PYZus{}mf\PYZus{}train}\PY{p}{,}\PY{n}{y\PYZus{}mf\PYZus{}train\PYZus{}norm}\PY{p}{,}\PY{n}{X\PYZus{}mf\PYZus{}test}\PY{p}{)}
             \PY{n}{rmse} \PY{o}{=} \PY{n}{np}\PY{o}{.}\PY{n}{sqrt}\PY{p}{(}
                 \PY{n}{mean\PYZus{}squared\PYZus{}error}\PY{p}{(}\PY{n}{scaler}\PY{o}{.}\PY{n}{inverse\PYZus{}transform}\PY{p}{(}\PY{n}{y\PYZus{}pred}\PY{o}{.}\PY{n}{reshape}\PY{p}{(}\PY{o}{\PYZhy{}}\PY{l+m+mi}{1}\PY{p}{,} \PY{l+m+mi}{1}\PY{p}{)}\PY{p}{)}\PY{p}{,}\PY{n}{y\PYZus{}mf\PYZus{}test}\PY{p}{)}\PY{p}{)}
             \PY{n}{rmse\PYZus{}test}\PY{o}{.}\PY{n}{append}\PY{p}{(}\PY{n}{rmse}\PY{p}{)}
\end{Verbatim}


    \begin{Verbatim}[commandchars=\\\{\}]
/Users/apple/.pyenv/versions/anaconda3-2.5.0/lib/python3.5/site-packages/sklearn/utils/validation.py:429: DataConversionWarning: Data with input dtype int64 was converted to float64 by StandardScaler.
  warnings.warn(msg, \_DataConversionWarning)

    \end{Verbatim}

    \begin{Verbatim}[commandchars=\\\{\}]
{\color{incolor}In [{\color{incolor}21}]:} \PY{c+c1}{\PYZsh{} RMSEのプロット}
         \PY{n}{ind} \PY{o}{=} \PY{n}{np}\PY{o}{.}\PY{n}{arange}\PY{p}{(}\PY{n+nb}{len}\PY{p}{(}\PY{n}{rmse\PYZus{}test}\PY{p}{)}\PY{p}{)}
         \PY{n}{bar} \PY{o}{=} \PY{n}{plt}\PY{o}{.}\PY{n}{bar}\PY{p}{(}\PY{n}{ind}\PY{p}{,}\PY{n}{height}\PY{o}{=}\PY{n}{rmse\PYZus{}test}\PY{p}{)}
         \PY{n}{plt}\PY{o}{.}\PY{n}{xticks}\PY{p}{(}\PY{n}{ind}\PY{p}{,}\PY{p}{[}\PY{l+s+s2}{\PYZdq{}}\PY{l+s+s2}{A}\PY{l+s+s2}{\PYZdq{}}\PY{p}{,}\PY{l+s+s2}{\PYZdq{}}\PY{l+s+s2}{B}\PY{l+s+s2}{\PYZdq{}}\PY{p}{,}\PY{l+s+s2}{\PYZdq{}}\PY{l+s+s2}{C}\PY{l+s+s2}{\PYZdq{}}\PY{p}{,}\PY{l+s+s2}{\PYZdq{}}\PY{l+s+s2}{D}\PY{l+s+s2}{\PYZdq{}}\PY{p}{,}\PY{l+s+s2}{\PYZdq{}}\PY{l+s+s2}{E}\PY{l+s+s2}{\PYZdq{}}\PY{p}{]}\PY{p}{)}
         \PY{n}{plt}\PY{o}{.}\PY{n}{ylim}\PY{p}{(}\PY{p}{(}\PY{l+m+mf}{0.88}\PY{p}{,}\PY{l+m+mf}{0.90}\PY{p}{)}\PY{p}{)}
\end{Verbatim}


\begin{Verbatim}[commandchars=\\\{\}]
{\color{outcolor}Out[{\color{outcolor}21}]:} (0.88, 0.9)
\end{Verbatim}
            
    \begin{center}
    \adjustimage{max size={0.9\linewidth}{0.9\paperheight}}{output_24_1.png}
    \end{center}
    { \hspace*{\fill} \\}
    
    \begin{itemize}
\tightlist
\item
  上記結果より、(ユーザーID・映画ID・性別・評価年)の組み合わせが良いことがわかる
\item
  次に、上記組み合わせの中で、適当なrankの値を選択する
\end{itemize}

    \begin{Verbatim}[commandchars=\\\{\}]
{\color{incolor}In [{\color{incolor}23}]:} \PY{c+c1}{\PYZsh{} 最も良い特徴量の組み合わせ}
         \PY{n}{best\PYZus{}column} \PY{o}{=} \PY{p}{[}\PY{l+s+s2}{\PYZdq{}}\PY{l+s+s2}{user\PYZus{}id}\PY{l+s+s2}{\PYZdq{}}\PY{p}{,}\PY{l+s+s2}{\PYZdq{}}\PY{l+s+s2}{movie\PYZus{}id}\PY{l+s+s2}{\PYZdq{}}\PY{p}{,}\PY{l+s+s2}{\PYZdq{}}\PY{l+s+s2}{sex}\PY{l+s+s2}{\PYZdq{}}\PY{p}{,}\PY{l+s+s2}{\PYZdq{}}\PY{l+s+s2}{year}\PY{l+s+s2}{\PYZdq{}}\PY{p}{,}\PY{l+s+s2}{\PYZdq{}}\PY{l+s+s2}{rating}\PY{l+s+s2}{\PYZdq{}}\PY{p}{]}
         \PY{c+c1}{\PYZsh{} rankの候補}
         \PY{n}{ranks} \PY{o}{=} \PY{p}{[}\PY{l+m+mi}{4}\PY{p}{,}\PY{l+m+mi}{8}\PY{p}{,}\PY{l+m+mi}{16}\PY{p}{,}\PY{l+m+mi}{32}\PY{p}{,}\PY{l+m+mi}{64}\PY{p}{]}
\end{Verbatim}


    \begin{Verbatim}[commandchars=\\\{\}]
{\color{incolor}In [{\color{incolor}24}]:} \PY{n}{rmse\PYZus{}test} \PY{o}{=} \PY{p}{[}\PY{p}{]}
         \PY{k}{for} \PY{n}{rank} \PY{o+ow}{in} \PY{n}{ranks}\PY{p}{:}
             \PY{c+c1}{\PYZsh{} 欠損値削除}
             \PY{n}{filtered\PYZus{}lens} \PY{o}{=} \PY{n}{lens}\PY{p}{[}\PY{n}{best\PYZus{}column}\PY{p}{]}\PY{o}{.}\PY{n}{dropna}\PY{p}{(}\PY{p}{)}
             \PY{c+c1}{\PYZsh{} 入力データをダミー変数に変換}
             \PY{n}{v} \PY{o}{=} \PY{n}{DictVectorizer}\PY{p}{(}\PY{p}{)}
             \PY{n}{X\PYZus{}more\PYZus{}feature} \PY{o}{=} \PY{n}{v}\PY{o}{.}\PY{n}{fit\PYZus{}transform}\PY{p}{(}
                 \PY{n+nb}{list}\PY{p}{(}\PY{n}{filtered\PYZus{}lens}\PY{o}{.}\PY{n}{drop}\PY{p}{(}\PY{l+s+s2}{\PYZdq{}}\PY{l+s+s2}{rating}\PY{l+s+s2}{\PYZdq{}}\PY{p}{,}\PY{n}{axis}\PY{o}{=}\PY{l+m+mi}{1}\PY{p}{)}\PY{o}{.}\PY{n}{T}\PY{o}{.}\PY{n}{to\PYZus{}dict}\PY{p}{(}\PY{p}{)}\PY{o}{.}\PY{n}{values}\PY{p}{(}\PY{p}{)}\PY{p}{)}\PY{p}{)}
             \PY{n}{y\PYZus{}more\PYZus{}feature} \PY{o}{=} \PY{n}{filtered\PYZus{}lens}\PY{p}{[}\PY{l+s+s2}{\PYZdq{}}\PY{l+s+s2}{rating}\PY{l+s+s2}{\PYZdq{}}\PY{p}{]}\PY{o}{.}\PY{n}{tolist}\PY{p}{(}\PY{p}{)}
             
             \PY{c+c1}{\PYZsh{} 教師データの学習用と評価用の分割}
             \PY{n}{X\PYZus{}mf\PYZus{}train}\PY{p}{,}\PY{n}{X\PYZus{}mf\PYZus{}test}\PY{p}{,}\PY{n}{y\PYZus{}mf\PYZus{}train}\PY{p}{,}\PY{n}{y\PYZus{}mf\PYZus{}test} \PY{o}{=} \PY{n}{train\PYZus{}test\PYZus{}split}\PY{p}{(}
                 \PY{n}{X\PYZus{}more\PYZus{}feature}\PY{p}{,}\PY{n}{y\PYZus{}more\PYZus{}feature}\PY{p}{,}\PY{n}{test\PYZus{}size}\PY{o}{=}\PY{o}{.}\PY{l+m+mi}{1}\PY{p}{,}\PY{n}{random\PYZus{}state}\PY{o}{=}\PY{l+m+mi}{42}\PY{p}{)}
             
             \PY{c+c1}{\PYZsh{} ratingの正規化}
             \PY{n}{scaler} \PY{o}{=} \PY{n}{StandardScaler}\PY{p}{(}\PY{p}{)}
             \PY{n}{y\PYZus{}mf\PYZus{}train\PYZus{}norm} \PY{o}{=} \PY{n}{scaler}\PY{o}{.}\PY{n}{fit\PYZus{}transform}\PY{p}{(}\PY{n}{np}\PY{o}{.}\PY{n}{array}\PY{p}{(}\PY{n}{y\PYZus{}mf\PYZus{}train}\PY{p}{)}\PY{o}{.}\PY{n}{reshape}\PY{p}{(}\PY{o}{\PYZhy{}}\PY{l+m+mi}{1}\PY{p}{,} \PY{l+m+mi}{1}\PY{p}{)}\PY{p}{)}\PY{o}{.}\PY{n}{ravel}\PY{p}{(}\PY{p}{)}
             
             \PY{c+c1}{\PYZsh{} MCMCを使ったモデルの学習}
             \PY{n}{fm} \PY{o}{=} \PY{n}{mcmc}\PY{o}{.}\PY{n}{FMRegression}\PY{p}{(}\PY{n}{n\PYZus{}iter}\PY{o}{=}\PY{l+m+mi}{500}\PY{p}{,}\PY{n}{rank}\PY{o}{=}\PY{n}{rank}\PY{p}{,}\PY{n}{random\PYZus{}state}\PY{o}{=}\PY{l+m+mi}{123}\PY{p}{)}
             \PY{n}{fm}\PY{o}{.}\PY{n}{fit\PYZus{}predict}\PY{p}{(}\PY{n}{X\PYZus{}mf\PYZus{}train}\PY{p}{,}\PY{n}{y\PYZus{}mf\PYZus{}train\PYZus{}norm}\PY{p}{,}\PY{n}{X\PYZus{}mf\PYZus{}test}\PY{p}{)}
             
             \PY{c+c1}{\PYZsh{} テストデータでの予測結果のRMSE取得}
             \PY{n}{y\PYZus{}pred} \PY{o}{=} \PY{n}{fm}\PY{o}{.}\PY{n}{fit\PYZus{}predict}\PY{p}{(}\PY{n}{X\PYZus{}mf\PYZus{}train}\PY{p}{,}\PY{n}{y\PYZus{}mf\PYZus{}train\PYZus{}norm}\PY{p}{,}\PY{n}{X\PYZus{}mf\PYZus{}test}\PY{p}{)}
             \PY{n}{rmse} \PY{o}{=} \PY{n}{np}\PY{o}{.}\PY{n}{sqrt}\PY{p}{(}
                 \PY{n}{mean\PYZus{}squared\PYZus{}error}\PY{p}{(}\PY{n}{scaler}\PY{o}{.}\PY{n}{inverse\PYZus{}transform}\PY{p}{(}\PY{n}{y\PYZus{}pred}\PY{o}{.}\PY{n}{reshape}\PY{p}{(}\PY{o}{\PYZhy{}}\PY{l+m+mi}{1}\PY{p}{,} \PY{l+m+mi}{1}\PY{p}{)}\PY{p}{)}\PY{p}{,}\PY{n}{y\PYZus{}mf\PYZus{}test}\PY{p}{)}\PY{p}{)}
             \PY{n}{rmse\PYZus{}test}\PY{o}{.}\PY{n}{append}\PY{p}{(}\PY{n}{rmse}\PY{p}{)}
             \PY{n+nb}{print}\PY{p}{(}\PY{l+s+s2}{\PYZdq{}}\PY{l+s+s2}{rank:}\PY{l+s+si}{\PYZob{}\PYZcb{}}\PY{l+s+se}{\PYZbs{}t}\PY{l+s+s2}{rmse:}\PY{l+s+si}{\PYZob{}:.3f\PYZcb{}}\PY{l+s+s2}{\PYZdq{}}\PY{o}{.}\PY{n}{format}\PY{p}{(}\PY{n}{rank}\PY{p}{,}\PY{n}{rmse}\PY{p}{)}\PY{p}{)}
\end{Verbatim}


    \begin{Verbatim}[commandchars=\\\{\}]
/Users/apple/.pyenv/versions/anaconda3-2.5.0/lib/python3.5/site-packages/sklearn/utils/validation.py:429: DataConversionWarning: Data with input dtype int64 was converted to float64 by StandardScaler.
  warnings.warn(msg, \_DataConversionWarning)

    \end{Verbatim}

    \begin{Verbatim}[commandchars=\\\{\}]
rank:4	rmse:0.889
rank:8	rmse:0.885
rank:16	rmse:0.885
rank:32	rmse:0.886
rank:64	rmse:0.887

    \end{Verbatim}

    \begin{Verbatim}[commandchars=\\\{\}]
{\color{incolor}In [{\color{incolor}25}]:} \PY{c+c1}{\PYZsh{} 結果のプロット}
         \PY{n}{plt}\PY{o}{.}\PY{n}{plot}\PY{p}{(}\PY{n}{ranks}\PY{p}{,}\PY{n}{rmse\PYZus{}test}\PY{p}{,}\PY{n}{label}\PY{o}{=}\PY{l+s+s2}{\PYZdq{}}\PY{l+s+s2}{dev test rmse}\PY{l+s+s2}{\PYZdq{}}\PY{p}{,}\PY{n}{color}\PY{o}{=}\PY{l+s+s2}{\PYZdq{}}\PY{l+s+s2}{r}\PY{l+s+s2}{\PYZdq{}}\PY{p}{)}
         \PY{n}{plt}\PY{o}{.}\PY{n}{legend}\PY{p}{(}\PY{p}{)}
\end{Verbatim}


\begin{Verbatim}[commandchars=\\\{\}]
{\color{outcolor}Out[{\color{outcolor}25}]:} <matplotlib.legend.Legend at 0x11b8a1240>
\end{Verbatim}
            
    \begin{center}
    \adjustimage{max size={0.9\linewidth}{0.9\paperheight}}{output_28_1.png}
    \end{center}
    { \hspace*{\fill} \\}
    
    \begin{itemize}
\tightlist
\item
  僅差ではあるが、rank=8の時に最もRMSEが小さい
\end{itemize}

    \begin{Verbatim}[commandchars=\\\{\}]
{\color{incolor}In [{\color{incolor}27}]:} \PY{c+c1}{\PYZsh{} 検証データの標準偏差(=常に平均値を出力するモデルのRMSE。モデルの良さのベンチマーク)}
         \PY{n}{np}\PY{o}{.}\PY{n}{std}\PY{p}{(}\PY{n}{y\PYZus{}more\PYZus{}feature}\PY{p}{)}
\end{Verbatim}


\begin{Verbatim}[commandchars=\\\{\}]
{\color{outcolor}Out[{\color{outcolor}27}]:} 1.1256679707622492
\end{Verbatim}
            
    \begin{itemize}
\tightlist
\item
  Factorization
  MachinesのRMSEはテストデータの標準偏差1.13を下回っている。\\
  つまり、常に平均値を出力する予測モデルより良いモデルが作成できたといえる
\end{itemize}

    \subsection{備考}\label{ux5099ux8003}


    % Add a bibliography block to the postdoc
    
    
    
    \end{document}
